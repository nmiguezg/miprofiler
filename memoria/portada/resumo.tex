%%%%%%%%%%%%%%%%%%%%%%%%%%%%%%%%%%%%%%%%%%%%%%%%%%%%%%%%%%%%%%%%%%%%%%%%%%%%%%%%

\pagestyle{empty}
\begin{abstract}

    En una era donde las redes sociales se han convertido en el espacio central para la difusión de la información, las interacciones personales y el debate sobre asuntos de actualidad, técnicas para el análisis de datos masivos como el perfilado automático de usuarios, nos brindan un visión más profunda y extensa de determinados fenómenos y procesos sociales; permitiéndonos razonar más precisamente acerca las causas y protagonistas de los mismos. En concreto, nuestro objetivo se centra en razonar acerca del movimiento \textit{Black Lives Matter} (\#BLM), a partir de una colección de referencia, que contiene publicaciones de usuarios procedentes de redes sociales que debatían sobre esta en el momento de máximas tensiones de la misma.
    
    Con esto en mente, mediante el seguimiento de un enfoque iterativo e incremental basado en una metodología ágil como Scrum, se ha construido una plataforma que permita inferir características personales, como edad y género, acerca de grandes colecciones de usuarios de manera eficiente y precisa. Para ello, se ha hecho un estudio previo acerca de los métodos del estado del arte, con mejor rendimiento actual, en materia de perfilado automático de usuarios, en idioma español. Además, esta incluye el desarrollo de una aplicación web a modo de \textit{dashboard} accesible e intuitivo, para la visualización y el análisis de los resultados obtenidos de aplicar estos algoritmos a la colección de referencia. 
    
    Finalmente, se presentan los resultados del perfilado de la colección de referencia, obtenidos mediante las plataforma desarrollada y se hace una reflexión acerca de los mismos identificando sus posibles causas, en relación a la naturaleza de la misma.

  \vspace*{25pt}
  \begin{segundoresumo}

In an era where social media has become the central space for the dissemination of information, personal interactions, and discussions on current issues, techniques for the analysis of massive data, such as automatic user profiling, provide us with a deeper and broader insight into specific social phenomena and processes, allowing us to reason more precisely about their causes and protagonists. Specifically, our focus is on reasoning about the \textit{Black Lives Matter} movement (\#BLM) based on a reference collection containing user posts from social media that discussed it during its peak tensions.

With this in mind, following an iterative and incremental approach based on an agile methodology like Scrum, a platform has been built to efficiently and accurately infer personal characteristics, such as age and gender, for large user collections. For this purpose, a previous study has been conducted on the state-of-the-art methods with the best current performance in automatic author profiling in the Spanish language. Additionally, this includes the development of a web application in the form of an accessible and intuitive dashboard for the visualization and analysis of the results obtained by applying these algorithms to the reference collection.

Finally, the results of profiling the reference collection obtained through the developed platform are presented, and a reflection is made on them, identifying their possible causes in relation to the nature of the collection.

% In an age where social media has taken center stage for information dissemination, personal interactions, and discussions on contemporary topics, advanced data analysis techniques, such as automatic user profiling, provide us with a deeper and more comprehensive understanding of specific social phenomena and processes. This enhanced comprehension enables us to more accurately discern their underlying causes and key figures. To illustrate, our primary focus centers on dissecting the "Black Lives Matter" movement (#BLM) using a reference dataset consisting of user-generated content from social media during the height of its tensions.

% With this perspective in mind, we have adopted an iterative and incremental approach rooted in agile methodologies like Scrum to construct a platform that efficiently and precisely deduces personal attributes, including age and gender, for extensive user datasets. This endeavor involves an initial investigation into the latest state-of-the-art techniques in automatic user profiling within the Spanish language context. Furthermore, it encompasses the development of a user-friendly web application serving as an accessible and intuitive dashboard, facilitating the visualization and analysis of the results derived from the application of these algorithms to the reference dataset.

% In conclusion, the findings from profiling the reference dataset, acquired through the utilization of the developed platform, are presented, followed by a reflection that delves into potential causal factors in relation to the dataset's intrinsic nature.
  \end{segundoresumo}
\vspace*{25pt}
\begin{multicols}{2}
\begin{description}
\item [\palabraschaveprincipal:] \mbox{} \\[-20pt]
%poner como 7 items
\begin{itemize}
    \item Perfilado automático
    \item Redes Sociales
    \item Archivos sociales
    \item Black Lives Matter
    \item Aprendizaje automático
    \item Aplicación web
    \item Procesado de lenguaje natural
    \item Aprendizaje profundo
\end{itemize}
\end{description}
\begin{description}
\item [\palabraschavesecundaria:] \mbox{} \\[-20pt]
  \blindlist{itemize}[7] % substitúe este comando por un itemize
                         % que relacione as palabras chave
                         % que mellor identifiquen o teu TFG
                         % no idioma secundario da memoria (tipicamente: inglés)
\end{description}
\end{multicols}

\end{abstract}
\pagestyle{fancy}

%%%%%%%%%%%%%%%%%%%%%%%%%%%%%%%%%%%%%%%%%%%%%%%%%%%%%%%%%%%%%%%%%%%%%%%%%%%%%%%%
