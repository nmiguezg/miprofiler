%%%%%%%%%%%%%%%%%%%%%%%%%%%%%%%%%%%%%%%%%%%%%%%%%%%%%%%%%%%%%%%%%%%%%%%%%%%%%%%%
% Obxectivo: Lista de termos empregados no documento,                          %
%            xunto cos seus respectivos significados.                          %
%%%%%%%%%%%%%%%%%%%%%%%%%%%%%%%%%%%%%%%%%%%%%%%%%%%%%%%%%%%%%%%%%%%%%%%%%%%%%%%%

\newglossaryentry{bytecode}{
  name=bytecode,
  description={Código independente da máquina que xeran compiladores de determinadas linguaxes (Java, Erlang,\dots) e que é executado polo correspondente intérprete.}
}
\newglossaryentry{author_analysis}{
    name=\textit{authorship analysis},
    description={}
}
\newglossaryentry{datasets}{
    name=\textit{datasets},
    description={}
}
\newglossaryentry{svm}{
    name=SVM,
    description={Máquina de soporte vectorial}
}
\newglossaryentry{lr}{
    name=LR,
    description={Logistic regression}
}
\newglossaryentry{mlp}{
    name=MLP,
    description={Multi-layer perceptron}
}
\newglossaryentry{em}{
    name=EM,
    description={(Ensemble models): combinación de modelos de aprendizaje automático para obtener un mejor rendimiento que con el uso solamente de alguno de los modelos constituyentes.}
}
\newglossaryentry{end2end}{
    name=pruebas end-to-end,
    description={Es una metodología de pruebas software que consiste en probar el producto de extremo a extremos, es decir, desde la perspectiva del usuario final. Sirve para detectar fallos en la interfaz de usuario.
    }
}
\newglossaryentry{web_scraping}{
    name=web scraping,
    description={El web scraping se refiere a la extracción automatizada mediante software de datos e información contenidos en páginas web.}
}
\newglossaryentry{mockups}{
    name=\textit{mockups},
    description={Los mockups son representaciones visuales simples y esquemáticas de interfaces de usuario que se realizan en la etapa de análisis de requisitos de un proyecto, que tienen como objetivo comunicar y visualizar la estructura y comportamiento de un producto antes de pasar a la fase de desarrollo.}
}