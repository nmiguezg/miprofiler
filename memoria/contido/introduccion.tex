\chapter{Introducción}
\label{chap:introduccion}
\lettrine{E}{n} este capítulo se expone el contexto, motivación y objetivos principales del proyecto así como la estructura de la memoria.

\section{Motivación}

El surgimiento de las redes sociales en la década de los 2000, ha transformado significativamente el modo de vida de la población general. Estas se han convertido en el espacio por defecto para las comunicaciones personales, la compartición de fotos, la expresión de opiniones, estados de ánimo, historias, memes, etc. Llegando a ser uno de los medios preferidos para la difusión de información y el debate acerca de asuntos relevantes del momento. De hecho, el crecimiento de estas plataformas ha propiciado el incremento de temas de debate y actualidad que se dan en la sociedad, abarcando todas las ámbitos posibles: política, cultura, moda, ética, ocio, etc.

Por este motivo, numerosos investigadores se han dedicado a crear y estudiar archivos nativos digitales desde redes sociales \citep{acker_2014, pybus_2013}, de forma que estos puedan ser reusados posteriormente por otros investigadores para el estudio de determinados procesos y fenómenos sociales.

Uno de estos fenómenos de interés que se ha documentado a través de la creación de un archivo social es \acrfull{blm}. \acrlong{blm} ("las vidas negras importan" en español) es un movimiento descentralizado surgido en el año 2013 en Estados Unidos en respuesta a la violencia policial dirigida hacia las personas negras. Este busca acabar con la represión sistémica y el racismo estructural que afectan a las comunidades negras. Desde entonces se viene utilizando el lema para protestar contra los símbolos racistas contra la comunidad negra. No obstante, en mayo de 2020 fue cuando alcanzó mayor repercusión debido al asesinato de George Floyd. A partir de entonces, se originaron respuestas en forma de manifestaciones y protestas pacíficas, que también se materializaron en las redes sociales en forma de debates.\\
Gracias al trabajo de \citet{heritage_BLM}, que reunieron unas colecciones de referencia, en inglés\footnote{\url{https://www.dc.fi.udc.es/~david/heritage/}} y español\footnote{\url{https://www.dc.fi.udc.es/~david/hdh2021/}}, acerca de las protestas causadas debido al movimiento, ahora estas se pueden usar para estudiar dicho movimiento en forma de archivo social. Este archivo tiene un año de actividad desde la muerte de George Floyd, e incluye más de 260.000 posts de 90.000 usuarios recogidos de una red social llamada Reddit\footnote{\url{https://www.reddit.com/}}, que compartieron mensajes y contenido acerca de \acrshort{blm}.\\
% **Este párrafo siguiente iría más bien en objetivos
% Aquí lo suyo sería mencionar un poco el interés en saber quien comentaba en las colecciones anteriores**\\
Estas colecciones de referencia son muy útiles a la hora de observar las diversas opiniones de los usuarios acerca de este movimiento. Pudiendo llegar a realizar estudios sobre la posición mayoritaria de la gente con respecto al mismo o los argumentos que tienen mayor peso a la hora de apoyarlo. Sin embargo, un aspecto fundamental acerca de la colección del que no se tiene información explícita está relacionado con: ¿Quién publicaba en la colección? No sabemos nada acerca del perfil demográfico, generacional o rasgos psicológicos de los usuarios de la colección. Conociendo esta información podríamos indagar de manera más profunda en el estudio de la misma.

Se conoce como perfilado automático de usuarios al procesamiento automático de información, como publicaciones y textos procedentes de redes sociales, mediante técnicas de procesado de lenguaje natural e inteligencia artificial con el objetivo de extraer características, preferencias o comportamientos acerca de estos. En el pasado, diversos autores ya demostraron que es posible la extracción en base al estilo de escritura y lenguaje utilizado de ciertas características acerca de los autores de textos como pueden ser edad, género o rasgos de personalidad \citep{personality_profiling, words-nerds}. No obstante, el incremento por interés en el perfilado automático de usuarios en redes sociales se vio patente en los últimos años en la organización de una serie de competiciones y trabajos con el objetivo de mejorar los algoritmos existentes de perfilado automático en distintos idiomas \citep{pan:2015,iberlef2022,profiling_urdu}.

Este proyecto, por tanto surge como parte de los esfuerzos por intentar conocer en detalle aquellos perfiles que se dan en estas colecciones de referencia, sobre el movimiento \acrshort{blm} en redes sociales. Como parte de estos esfuerzos, se definieron dos proyectos en paralelo, uno sobre el subconjunto de las colecciones en idioma inglés y otro sobre el subconjunto español. Este en concreto, busca analizar los perfiles que se dan en la parte en español de estas colecciones, teniendo en cuenta los retos que esto conlleva debido a la falta de recursos en comparación con el idioma inglés.
% intentar sacar información de los perfiles que comentan en las colecciones como BLM
% Desde el inicio de las redes sociales, la anonimidad de los usuarios ha sido una característica fundamental de las mismas. En estas en muchas ocasiones la gente se crea seudónimos que sirven para ocultar su verdadera identidad, de modo que les permiten escribir abiertamente de ciertos temas de los que normalmente no se atreverían a hablar con ciertos conocidos. Esta anonimización por parte de los usuarios se puede ver en algunas redes más que otras: Reddit por ejemplo es una red social en la que apenas se comparte información personal y en la que la gente suele tener seudónimo como nombre de usuario, a diferencia de otras como Facebook donde es más común compartir información personal. \\
% Por ello, el interés por averiguar quién está detrás de los usuarios que escriben en redes sociales ha llevado a investigadores intentar sacar información acerca de los usuarios que usan estas redes sociales. Se denomina perfilado automático de usuarios al proceso de análisis de datos e inteligencia artificial en el que se busca crear perfiles detallados de los usuarios usando la información disponible acerca de ellos
% En los últimos años las redes sociales se han ido convirtiendo poco a poco en el medio principal de difusión de la cultura y discusión de todo tipo de temas de actualidad. Desde política, deportes, moda, cine, televisión, literatura. 
\section{Objetivos}
El propósito de este trabajo es por tanto estudiar las anteriores colecciones desde la perspectiva del usuario. Esto es: se pretende obtener información de las distintas personas que interactuaban y debatían sobre \acrshort{blm} en redes sociales. De manera que, se puedan analizar las distintas opiniones de los usuarios en función de sus características demográficas, ideológicas y de personalidad.

Como hemos comentado anteriormente, este trabajo se centrará completamente en la colección de los usuarios que escribían en idioma español. Para ello, será necesario revisar y analizar las técnicas del estado del arte en materia de perfilado en este idioma, con la intención de replicarlas para el estudio de nuestra colección.

Como resultado, a continuación se exponen en detalle los objetivos identificados en este proyecto:

\begin{itemize}
    \item Analizar y evaluar los algoritmos y modelos existentes del estado del arte en el ámbito del perfilado automático  de usuarios, especialmente en materia de edad y género, en redes sociales como Reddit o Twitter.
    \item Seleccionar algunos de los algoritmos analizados anteriormente con el objetivo de aplicarlos a las colecciones de \acrshort{blm}.
    \item Estudiar los resultados obtenidos de aplicar estos algoritmos, a partir de la creación de una aplicación web a modo de \textit{dashboard} que muestre la distribución de usuarios, que publicaban en las colecciones, en función de sus características demográficas.
    % \item Aplicar las mejoras que se consideren oportunas a los algoritmos anteriores con el objetivo de extraer otras características distintas acerca de la colección.
\end{itemize}

\section{Estructura de la memoria}
A continuación, se describe la organización de la estructura de la memoria del presente trabajo. Esta se encuentra dividida en los siguientes 9 capítulos:

\begin{enumerate}

    \item \hyperref[chap:introduccion]{\textit{\textbf{Introducción}}}: capítulo actual donde se enuncian los aspectos principales del proyecto como motivación y objetivos del mismo. Asimismo, también se expone la estructura global del presente documento.
    
    \item \hyperref[chap:fundamentos]{\textit{\textbf{Fundamentos}}}: este capítulo corresponde a la fase del trabajo donde se evalúa y analiza el estado del arte correspondiente a las técnicas y algoritmos de perfilado automático que se aplicarán a la colección de \acrshort{blm}.
    
    \item \hyperref[chap:tecnologias]{\textit{\textbf{Tecnologías y herramientas utilizadas}}}: este capítulo presenta las tecnologías y herramientas que puedan ser relevante mencionar, empleadas en alguna de las fases del proyecto.
    
    \item \hyperref[chap:metodologia]{\textit{\textbf{Metodología y gestión del proyecto}}}: en él se explica el enfoque metodológico empleado para la realización del proyecto, al igual que la planificación seguida.

    \item \hyperref[chap:analisis]{\textit{\textbf{Análisis}}}: en este capítulo se muestran los requisitos tanto funcionales como no funcionales que debe cumplir la herramienta desarrollada.
    
    \item \hyperref[chap:design]{\textit{\textbf{Diseño}}}: en este capítulo se exponen las decisiones de diseño, en cuanto a arquitectura e interfaz de usuario tomadas para el desarrollo de la herramienta construida.
    
    \item \hyperref[chap:desarrollo]{\textit{\textbf{Desarrollo}}}: aquí se describen los incrementos realizados en cada iteración del proyecto, así como el flujo de trabajo seguido.

    \item \hyperref[chap:blm]{\textit{\textbf{Análisis colección \acrshort{blm}}}}: aquí se muestran los resultados de aplicar los algoritmos de perfilado seleccionados a la colección de \acrshort{blm}. Estos se muestra a través de un \textit{dashboard} web.

    \item \hyperref[chap:conclusiones]{\textit{\textbf{Conclusiones}}}: en este capítulo se exponen las conclusiones que se derivan del trabajo realizado respecto a la situación final del proyecto, inluyendo lecciones aprendidas durante el desarrolo del mismo y posibles líneas de trabajo futuras.
    
\end{enumerate}