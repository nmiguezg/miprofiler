\chapter{Desarrollo}
\label{chap:desarrollo}
\section{Sprints}
\subsection{Sprint 0: 8-2-2023 a 14-3-2023}
:
\begin{itemize}
    \item Configuración del entorno: búsqueda y experimentación con librerías de procesado del lenguaje natural como Spacy o NLTK
    \item Introducción a estado del arte del perfilado automático de usuarios en edad y género.
\end{itemize}

\subsection{Sprint 1: 14-3-2023 a 29-3-2023}
:
\begin{itemize}
    \item Experimentando técnicas de preprocesado de los textos de la colección como: lematización, sustitución de emojis por sus significado, etc.
    \item Investigación y búsqueda de conjuntos de entrenamiento en idioma español para la tarea de perfilado automático de usuarios.
    \item Estudio del lenguaje empleado en el corpus BLM con el objetivo de
    ***** meter la cabeza en el problema*****
\end{itemize}

\subsection{Sprint 2: 29-3-2023 a 17-4-2023}
:
\begin{itemize}
    \item Búsqueda, comparación y selección de algoritmos de perfilado automático
    \item Selección de conjuntos de entrenamiento a usar para entrenar los modelos seleccionados.
\end{itemize}

\subsection{Sprint 3: 17-4-2023 a 20-6-2023}%este alomejor hay que dividirlo en 2, no sé...
:
\begin{itemize}
    \item Carga de la colección BLM, agrupación de los usuarios y preprocesado de los posts.
    \item Adaptación y ejecución de pruebas con algoritmo LosCalis. Investigación de malos resultados obtenidos e intento de realización de mejoras: cambios de preprocesado, hiperparámetros, asignación de pesos a salidas en la red.
    \item Creación de scraper de twitter para poblar los ficheros que constituyen el dataset de 2016.
\end{itemize}

\subsection{Sprint 4: 20-6-2023 a 20-7-2023}
:
\begin{itemize}
    \item Ejecución en batches del scraper para descargar la colección entera.
    \item Adaptación y ejecución de algoritmos de modaresi y grivas.
    \item Realización de pruebas con estos (pruebas con distintos parámetros, features, preprocesado).
    \item Documentación de resultados y redacción de los dos primeros capítulos de la memoria  (\ref{chap:introducion} y \ref{chap:fundamentos}).
\end{itemize}
\subsection{Sprint 5: 20-7-2023 a 5-8-2023}
\begin{itemize}
    \item Inicio del backend: creación del endpoint, entrenamiento y guardado de algoritmos de perfilado para producción.
    \item Diseño arquitectura backend.
    \item Containerización
\end{itemize}
\subsection{Sprint 6: 6-8-2023 a 21-8-2023}
\begin{itemize}
    \item Desarrollo modelo base de datos.
    \item Diseño e implementación del frontend.
\end{itemize}
\subsection{Sprint 7: 21-8-2023 a 11-8-2023}
\begin{itemize}
    \item Como usuario quiero poder ver los distintos datos demográficos de los usuarios de la colección en forma de lista.
    \item Como usuario quiero ver en detalle los posts que ha publicado un usuario concreto de la colección.
    \item Estilado de elementos del frontend.
\end{itemize}